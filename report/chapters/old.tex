\chapter{First chapter}

\section{Some section}




\chapter{Some {\LaTeX} tip and tricks}

\section{Producing PDF-documents}

In some LaTeX distributions there exist a program called
\texttt{pdflatex}. This program makes the production of good looking PDF 
documents easy. Use it like you use \texttt{latex}. The difference is that
this program does not produce a dvi-file, but instead makes a
PDF-document directly. You should be careful with graphics, avoid
using the pstricks package if you like to use \texttt{pdflatex} and also
read the following section about image inclusion.

If \texttt{pdflatex} of some reason does not work for you, you could use
ps2pdf instead. It does not do as good job as pdflatex, and
you have to be careful with the input flags, otherwise the
PDF-document will look horrible, but you might have less trouble
getting your latex documents though the compiler.

\texttt{ps2pdf}, as the name says, converts PS-files to PDF-documents. So,
first run latex, then dvips, and finaly ps2pdf. Notice the flags
used below, they makes sure, the correct fonts are used. 
\begin{verbatim}
  > latex thesis.tex
  > dvips -D 600 -Z -G0 -Ppdf thesis.dvi | thesis.ps
  > ps2pdf thesis.ps
\end{verbatim}

\section{Including pictures}

One of the pros of {\LaTeX} is that it produces very good looking
documents, and therefore you also want the graphics to look nice. If
you have found a program that can produce vector graphics, use
it. It can often convert the pictures to the desired formats. 

However, graphics can be a mess. Graphics and {\LaTeX} is no 
exception. In this package, the \texttt{graphicx} package is used 
as it provides some good functionality. Unfortunately {\LaTeX} is a bit 
tricky about which graphics  formats to use. If you use 
\texttt{pdflatex} PDF-files and PNG-files can be used but with 
standard \texttt{latex} only EPS-files works (EPS stands for
Encapsulated PostScript). Both EPS and PDF are a good format for 
vector graphics but not so good for bitmaps. 

So, if you both like to produce PS as well as PDF documents, you
often need two versions of the pictures. This is probably not a
problem when they are easily exported to the desired format in the
image-software used (Illustrator, XFig etc.). 

But how to make {\LaTeX} understand which picture to use? Well, 
{\LaTeX} does this automagicaly. First we can tell {\LaTeX} where to 
find our pictures,
\begin{verbatim}
    \graphicspath{{pictures/}}
\end{verbatim} 
This makes {\LaTeX} search the subfolder \texttt{./pictures/} for 
pictures to include as well as the current folder.

Secondly, use the \verb+\includegraphics+ command to include your 
pictures. 

\begin{verbatim}
     \includegraphics[width=70mm]{complicated}
\end{verbatim}

Notice, that there are NO extension used on the name of the
picture. This makes {\LaTeX} search for the right one, e.g., a PNG-
file if it is a PDF document, otherwise an EPS-file.

You can do all sorts of tricks with the include pictures, where
scaling is the most common. Use 'scale', 'width', height' as in the
example above.

Also notice how the graphics package are included below if you are
not using this package.

\section{References}

There is a lot to be said about references. However. I recommend
that you find some information on how to use bibtex. It can be a bit
tricky at first but worth it any time in the long run. However, a
few things can be useful to notice.

Authors should always be given with their full names, with the sir
name last, e.g., Jan Pedher Johansson. If there are more then one
author they should ALL be separated with an 'and', regardless if
there are two, three or ten. This is all handeld and made correct
by bibtex. 
\begin{verbatim}
     Bo Andersson and Anders Eriksson and Erik Bosson  
\end{verbatim}

Company names, abbreviations etc. can look funny using bibtex. Use 
\{ \} if there are something you like to force.

\begin{tabular}{ll}
\texttt{\{Umeå Universitet\}} & (so it does not become U. Universitet) \\
\texttt{\{HTML\}} & (so it does not become html or Html)
\end{tabular}

In the report document class references are not included in the
table of contents. If you like it to be, use the trick described in
the previous section.

Remember to cite the literature and web-papers in an appropriate way. 
There are many ways to do that, such as the following \cite{Lamp94} where 
you can read about how to create a reference list in a {\LaTeX}\-report 
using BibTex.

\section{Other tricks}

Notice the \texttt{tabular*} and the \texttt{tabularx} (requires the 
tabularx package) environments. With these enviroments you can have tables 
with a fixed width: 
\begin{verbatim}
     \begin{tabular*}{0.8\linewidth}{rll}
     \end{tabular*}
\end{verbatim}
The \texttt{tabularx} has a great feature if you like a column that works
like a p{<width>}-column but with a variable width. In this example
the first column will stretch to use the full width of the table
depending on the with of the two last columns. 
\begin{verbatim}
     \begin{tabularx}{0.8\linewidth}{Xll}
     \end{tabularx}
\end{verbatim}
Notice that \texttt{tabularx} can not be used in your own defined
environments. 


In some large documents, it can look nice if the first pages, with
the abstract, possible acknowledgment, table of contents, list of
figures etc are numbered differently then the rest of the document,
usually with roman numbers. Then the numbering restarts with the
first chapter using normal numbers. To achieve this, put 
\begin{verbatim}
     \pagenumbering{roman}
\end{verbatim}
before the first page, then
\begin{verbatim}
     \cleardoublepage
     \setcounter{page}{1}
     \pagenumbering{arabic}
     \chapter{Introduction}
\end{verbatim}
before the first chapter.
