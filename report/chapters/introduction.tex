\def\cloudhistory{https://www.dataversity.net/brief-history-cloud-computing/}

\chapter{Introduction}
%%%%%%%%%
% Cloud %
%%%%%%%%%
In 2006 Amazon released its cloud platform AWS, which marks the start of large scale cloud computing. It allowed users to rent virtual machines to run internet-facing production services. Before the cloud became viable, companies needed to provide their infrastructure, making it harder and costlier to maintain and scale, especially for smaller players \cite{cost_analysis}. Over the years, more cloud platforms have emerged, cloud platforms has become more and more user-friendly. With new hosting solutions like the serverless design architecture\cite{serverless}, it is now possible to deploy an application directly on a cloud platform without requiring almost any operations knowledge.

%%%%%%%%%%%%%%%%%%%%
% Virtual Machines %
%%%%%%%%%%%%%%%%%%%%
However, cloud-based virtual machines are still relevant because of their broad use cases. Because they run a complete operating system\cite{vms_function}, they are configurable to run all possible applications and configurations. When running in a cloud platform, they also provide easy backup management and creation. However, virtual machines are not resource-efficient since they require virtualization of an entire operating system (OS) to function. It leads to slow creation, boot times and a significant performance overhead. 

%%%%%%%%%%%%%%
% Containers %
%%%%%%%%%%%%%%
Containers attempt to solve the problem of running an application differently. They do so by sharing their kernel with the host and only running the most necessary OS applications\cite{linux_containers}, resulting in a tiny footprint. An example is the alpine container image, a Linux container image that is less than 3MB
\footnote{Alpine. Alpine Container Image. \url{https://www.alpinelinux.org/downloads/}. Visited 2021-02-11}. Compared to its virtual machine counterpart, which is around 130MB\footnote{Alpine. Alpine Virtual Machine Image. \url{https://hub.docker.com/_/alpine}. Visited 2021-02-11}. Containers allow fast creations, boot times and a minimal performance overhead. 

%%%%%%%%%%%%%%%%%%%%
% OCI & Kubernetes %
%%%%%%%%%%%%%%%%%%%%
In enters the Open Container Initiative(OCI), an open-source specification for Linux containers\cite{oci_spec}. It enables the usage of different container runtimes and container build-tools. Modern web services often require to scale onto multiple nodes to maintain adequate performance and availability. Container orchestrators solve this by managing the network and scheduling of the running services\footnote{Kubernetes. Kubernetes. \url{https://kubernetes.io/}. Visited 2021-02-11.}. From a users perspective, a container orchestrator makes a cluster of nodes seem like a single entity. One commonly used container orchestrator is called Kubernetes. It is an open-source declaration based container orchestrator which handles all possible use cases. 

%%%%%%
% CI %
%%%%%%
A popular workflow in modern development is Continuous Integration (CI)\cite{continuous_integration}. In CI, it is common to run automatic tests on all committed code. If the tests pass, the application is built and pushed it to a registry. Running the CI jobs in a container orchestrator is often preferable\cite{ci_containers} when running a container-based workflow. In such a workflow, the built application is a container image, built by an OCI build-tool. These build-tools do not usually run well in containers unless given extended privileges, suboptimal for security reasons. 


\section{Aims and Objectives}
This thesis aims to investigate the field of container orchestrated OCI build-tools. It compares multiple different build-tools by performance, caching, cost, security and UX. They are evaluated by reading articles and papers and by running custom tests. The paper attempts to answer the following research questions:

\begin{enumerate}
    \item How securely can OCI container build-tools be run in a container orchestrator?
    \item How can the performance of multiple orchestrated container build-tools be tested?
    \item How do the most prominent build-tools compare from a performance and cache standpoint?
    \item What is the economic cost of running build-tools in production?
    \item How user-friendly is the most prominent OCI build-tools?
    \item How can a testing suite for OCI build-tools be developed?
\end{enumerate}

\subsection{Limitation}
This paper compares different orchestrated container build-tools. It will not compare how different container runtimes impact build-tools. One solution for running build-tools in a container is sysbox\cite{sysbox}. It is a container runtime which uses OS-virtualization to and userspace applications to make containers function as virtual machines. It allows software running in containers to run software like Systemd, Docker, and Kubernetes, which usually is impossible without changing some security features. 

\section{Background}
The case study for this thesis is Omegapoint. It is a Swedish IT consulting firm specialized in cybersecurity. They are moving a lot of their software infrastructure to the cloud and embracing a more container-based workflow. This thesis helps Omegapoint in deciding which OCI built-tool to use in container orchestrated environments. 

Omegapoint is an advanced consulting partner of Amazon and therefore provides credentials to Amazon's cloud platform AWS, the platform used to run the tests. Omegapoint also provides a sounding board and professional consultation for this thesis. 

\section{Risk Register}
When writing a thesis a lot of things can go wrong. It is therefore important to prevent and anticipate risks. Do this a risk register is created and available in table \ref{tab:risk_register}. It is used to mitigate risks and problems.

\begin{landscape}
\begin{table}[]
    \begin{tabular}{c|p{0.35\linewidth}|>{\centering}p{0.085\linewidth}|>{\centering}p{0.085\linewidth}|p{0.35\linewidth}}
    \textbf{\#} & \textbf{Risk}                               & \textbf{Likelihood \\ (L/M/H)} & \textbf{Impact\\(L/M/H)} & \textbf{Mitigation}                                         \\
    \hline
    1           & Configuring privileges is more time consuming than planned.     & L                           & H                       & Question supervisors for consultation.                                          \\ 
    \hline
    2           & Data corruption occurs.                                         & L                           & H                       & Use GIT for version control and backups                                         \\
    \hline
    3           & The specified build-tools cannot run in containers.             & L                           & H                       & When specifying build-tools, check if they can run within a container.          \\
    \hline
    4           & Getting access to a cloud platform is tougher than expected.    & L                           & H                       & Purchase it.                                                                    \\
    \hline
    5           & A supervisor becomes unresponsive.                              & L                           & L                       & The project is relatively self contained so local resources suffice.            \\
    \hline
    6           & The project get behind schedule.                                & L                           & H                       & Plan to finish earlier and work hard in the beginning.                          \\
    \hline
    7           & Finding published scientific papers is problematic.             & M                           & M                       & Identify other sources, including online technical reports and articles.        \\
    \hline
    8           & A large container base open source java project is not easily available. & L                  & L                       & Ask supervisors for help. If they cannot help, skip the real world test. \\
    \hline
    9           & Managing state with Terraform cloud is working as expected.     & L                           & L                       & Switch from remote state to local state.                                        \\
    \hline
    10          & The AWS container registry is slow.                             & L                           & M                       & Install the Harbor container registry onto the cluster to get a local one.      \\
    \hline
    11          & The pull limit on docker hub is reached.                        & L                           & L                       & Purchase docker premium.                                                        \\
    \hline
    12          & I get ill                                                       & L                           & H                       & Ask for project extension from the university supervisor                        \\
    \hline
    13          & If the university supervisor gets unresponsive.                 & L                           & H                       & Talk to the course coordinator to get a new one                                             
    \end{tabular}
    \centering
    \caption{Risk register}
    \label{tab:risk_register}
\end{table}
\end{landscape}

\section{Project Plan}
This thesis is a large project which spans four months, to make the writing and development more manageable, a project plan used. It is available in the form of a Gantt chart which viewable in Figure \ref{fig:gantt_chart}.

\definecolor{barblue}{RGB}{153,204,254}
\definecolor{groupblue}{RGB}{51,102,254}
\definecolor{linkred}{RGB}{165,0,33}
\renewcommand\sfdefault{phv}
\renewcommand\mddefault{mc}
\renewcommand\bfdefault{bc}
\setganttlinklabel{s-s}{START-TO-START}
\setganttlinklabel{f-s}{FINISH-TO-START}
\setganttlinklabel{f-f}{FINISH-TO-FINISH}
\sffamily

\begin{figure}
    \centering
    \begin{ganttchart}[
        canvas/.append style={fill=none, draw=black!5, line width=.75pt},
        hgrid style/.style={draw=black!5, line width=.75pt},
        vgrid={*1{draw=black!5, line width=.75pt}},
        today rule/.style={
          draw=black!64,
          dash pattern=on 3.5pt off 4.5pt,
          line width=1.5pt
        },
        y unit chart=0.8cm,
        today label font=\small\bfseries,
        title/.style={draw=none, fill=none},
        title label font=\bfseries\footnotesize,
        title label node/.append style={below=7pt},
        include title in canvas=false,
        bar label font=\mdseries\small\color{black!70},
        bar label node/.append style={left=2cm},
        bar/.append style={draw=none, fill=black!63},
        bar incomplete/.append style={fill=barblue},
        bar progress label font=\mdseries\footnotesize\color{black!70},
        group incomplete/.append style={fill=groupblue},
        group left shift=0,
        group right shift=0,
        group height=.5,
        group peaks tip position=0,
        group label node/.append style={left=.6cm},
        group progress label font=\bfseries\small,
        link/.style={-latex, line width=1.5pt, linkred},
        link label font=\scriptsize\bfseries,
        link label node/.append style={below left=-2pt and 0pt}
      ]{1}{17}
      \gantttitle[
        title label node/.append style={below left=7pt and -3pt}
      ]{WEEKS:\quad1}{1}
      \gantttitlelist{2,...,17}{1} \\
      
      \ganttgroup{Important dates}{1}{15} \\
      \ganttbar{Startup meeting}{1}{1} \\
      \ganttbar{Project plan}{3}{3} \\
      \ganttbar{Project report}{17}{17} \\
      
      \ganttgroup{Project Plan}{1}{3} \\
      \ganttbar{Writing}{1}{3} \\
      
      \ganttgroup{Report}{1}{15} \\
      \ganttbar{Introduction}{3}{3} \\
      \ganttbar{Literature survey}{4}{7} \\
      \ganttbar{Solution design}{8}{9} \\
      \ganttbar{Implementation}{9}{12} \\
      \ganttbar{Results}{10}{12} \\
      \ganttbar{Evaluation}{13}{14} \\
      \ganttbar{Conclusion}{15}{15} \\
      \ganttbar{Abstract}{15}{15} \\
      \ganttbar{Acknowledgements}{15}{15} \\
      
      \ganttgroup{Implementation}{9}{12} \\
      \ganttbar{Kubernetes cluster}{9}{9} \\
      \ganttbar{Deployment backend}{10}{10} \\
      \ganttbar{Web app}{11}{11} \\
      \ganttbar{Containerize}{12}{12} \\
      
      \ganttgroup{Extra time}{16}{17}
    \end{ganttchart}

    \caption{Project plan in the form of a Gantt chart}
    \label{fig:gantt_chart}
\end{figure}



\section{Thesis Structure}
The thesis is structured to start with a literature study where scientific papers, articles and manuals are studied. It does so by describing the fields of cloud, containers, DevOps, Kubernetes, OCI build tools and usability analysis. The chapter also sums the earlier work, which influences the rest of the thesis. 

It continues with a solution design which uses the conclusions of the literature study to create a solution. The chapter also specifies how to design different solution components. It displays the systems overall functionality and components on a high-level figure.

The thesis continues with an implementation chapter of the designed solution. It discusses how each component work at a low-level and other specifications of the implementation. It continues into a result chapter which presents the testing results. The chapter also explains the results and how to read them. 

An evaluation of the results is available in the next chapter. It discusses the impact of the results and concludes how the different build-tools performs. The chapter also recommends which build-tool that performs the best in the different test cases. The thesis ends with a conclusion section which discusses if and how the results are satisfactory. At last further work in the field is discussed, and a personal reflection of the project is presented. 
% Literature study
% Solution Design
% Implementation
% Results
% Evaluation
% Conclusion


% include what will be in the different chapters


%\def\cloudperks{\footnote{Salesforce UK. What are the Advantages of Cloud Computing? 10 Reasons to Move to the Cloud. Read 2021-01-18. \url{https://www.salesforce.com/uk/blog/2015/11/why-move-to-the-cloud-10-benefits-of-cloud-computing.html}}}
%\def\cloudhistory{\footnote{IBM Cloud Team. IBM. A Brief History of Cloud Computing. Read 2021-01-18. \url{https://www.ibm.com/cloud/blog/cloud-computing-history}}}
%\def\rhcontainers{\footnote{Red Hat. What's a Linux container?. Read 2021-01-18. \url{https://www.redhat.com/en/topics/containers/whats-a-linux-container}}}
%\def\ocicontainers{\footnote{OCI. About the Open Container Initiative. Read 2021-01-18. \url{https://opencontainers.org/about/overview/}}}
%\def\whatkubernetes{\footnote{Kubernetes. What is Kubernetes?. Read 2021-01-18. \url{https://kubernetes.io/docs/concepts/overview/what-is-kubernetes/}}}
